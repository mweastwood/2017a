% OUTLINE
% =======
%
% 1. Introduction
%
%   Here I will introduce the need for a new high-fidelity, low-frequency map of the sky. The
%   justifications are: 21 cm foreground mapping, galactic structure (?), others?
%
% 2. Observations
%
%   2.1 - Introduce the Owens Valley LWA
%   2.2 - Describe the 100 hour dataset
%   2.3 - Calibration strategy
%   2.4 - Source removal strategy
%   2.5 - Horizon RFI suppression strategy
%   2.6 - Beam fitting work
%   2.7 - Ionospheric conditions
%
% 3. Imaging
%
%   Give a brief overview of other widefield imaging algorithms. This should set the context for the
%   advantages of m-mode analysis.
%
%   3.1 - Introduce m-mode analysis imaging
%   3.2 - Describe my implementation of m-mode analysis on the ASTM
%   3.3 - L-curves (?)
%   3.4 - CLEAN
%
% 4. Results
%
%   Present sky maps with an estimate of the thermal noise in each pixel. Systematic errors will be
%   described in the following sub-sections.
%
%   4.1 - Horizon RFI
%   4.2 - Ionospheric effects
%   4.3 - Beam errors
%   4.4 - Peeling failures
%
% 5. Discussion
%
%   Compare with the Haslam map and Jayce's map.
%
% 6. Conclusion
%
% FIGURES
% =======
%
% * Antenna layout of the OVRO-LWA
% * Before/after snapshot image of the RFI removal
% * Image of the empirical beam (with source tracks)
% * Plot of ionospheric scintillation and refraction of an A-team source
% * Example L-curve
% * Zoom-in of the dirty PSF
% * Sky maps with their boot-strapped thermal noise estimates
% * (Model) image of the horizontal stripes that show up if the horizon RFI is not removed
% * (Model) image of the PSF distortion from ionospheric effects (scintillation, refraction)
% * (Model) image of the systematic errors in the map created from beam errors
% * (Model) image of the errors created from sporadic peeling failures

\documentclass[twocolumn]{aastex61}

\usepackage[T1]{fontenc}
\usepackage[utf8]{inputenc}
\usepackage{amsmath}
\usepackage[]{algorithm2e}

\renewcommand{\b}{\pmb}
\renewcommand{\d}{{\rm d}}
\newcommand{\atikh}{\b{\hat a}_\text{Tikhonov}}
\newcommand{\todo}[1]{\textcolor{red}{TODO: #1}\PackageWarning{TODO:}{#1!}}

\begin{document}

\title{The Radio Sky at Meter Wavelengths: $m$-Mode Analysis Imaging with the Owens Valley Long Wavelength Array}

\author{Michael W. Eastwood}
\affiliation{Department of Astronomy, California Institute of Technology, 1200 E California Blvd, Pasadena, CA 91125}

\author{Marin M. Anderson}
\affiliation{Department of Astronomy, California Institute of Technology, 1200 E California Blvd, Pasadena, CA 91125}

\author{Ryan Monroe}
\affiliation{Department of Electrical Engineering, California Institute of Technology, 1200 E California Blvd, Pasadena, CA 91125}

\author{Gregg Hallinan}
\affiliation{Department of Astronomy, California Institute of Technology, 1200 E California Blvd, Pasadena, CA 91125}

\author{Esayas B. Shume} % contributed ionosphere text
\affiliation{Jet Propulsion Laboratory, 4800 Oak Grove Dr, Pasadena, CA 91109}


\begin{abstract}
    I present all-sky maps of the sky from the Owens Valley Long Wavelength Array.
    These maps are created from the application of $m$-mode analysis. \todo{Wednesday}
\end{abstract}

\keywords{
    cosmology: observations --
    dark ages, reionization, first stars --
    radio continuum: galaxies --
    radio continuum: ISM
}

\section{Introduction}

Studies of the cosmic microwave background (CMB) have given us an unprecedented understanding of the
universe during the epoch of recombination \citep{2013ApJS..208...19H, 2014A&A...571A..16P,
2016A&A...594A..13P}. Before arriving at the Earth, the light from the CMB propagates through the
intergalactic medium (IGM), intervening galaxies and galaxy clusters, and the Milky Way's
interstellar medium (ISM). As photons from the CMB propagate through the intervening material
scattering, absorption, or additional emission can perturb the thermal CMB spectrum. In order to
measure the pristine CMB, these foreground components (eg. synchrotron emission, free-free emission,
thermal dust emission) are a nuisance and careful modeling and removal of these components is
essential \citep{2016A&A...594A..10P, 2016A&A...594A..25P}.

However in many cases the study of CMB foregrounds is scientifically interesting in its own right.
Measurements of dust-correlated anomalous microwave emission \citep{1997ApJ...486L..23L} led to the
discovery of spinning dust emission \citep{1998ApJ...508..157D}. Thomson scattering in the IGM
attenuates the CMB spectrum, but the optical depth implies a mean redshift to the Epoch of
Reionization (EoR) of $z_r = 8.8^{+1.7}_{-1.4}$ \citep{2016A&A...594A..13P}.  The thermal
Sunyaev-Zel'dovich effect \citep{1970CoASP...2...66S, 1972CoASP...4..173S} has been used to discover
new galaxy clusters \citep{2010ApJ...722.1180V} and measure the distance to galaxy clusters when
combined with X-ray measurements \citep{2006ApJ...647...25B}. The kinetic Sunyaev-Zel'dovich effect
has been used to constrain the duration of reionization $\Delta z_r \le 4.4$
\citep{2012ApJ...756...65Z}.

At redshifts $20 \gtrsim z \gtrsim 7$ the 21 cm hyperfine structure line of neutral hydrogen is
expected to produce a 10 to 100 mK perturbation in the CMB spectrum \citep{2006PhR...433..181F,
2012RPPh...75h6901P}. The amplitude of this perturbation on a given line-of-sight is a function of
the neutral fraction of hydrogen, the baryon overdensity, the spin temperature relative to the CMB
temperature at the given redshift, and the line-of-sight peculiar velocity of the gas.  The spatial
power spectrum of this perturbation is expected to be dominated by inhomogeneous heating of the IGM
at $z\sim 20$ \citep{2014MNRAS.437L..36F}, and by growing ionized bubbles during the EoR at $z\sim
7$ where a detection can constrain the ionizing efficiency of early galaxies, UV photon
mean-free-path, and the minimum halo mass that can support star formation
\citep{2015MNRAS.449.4246G}.

Current 21 cm cosmology experiments can be broadly separated into two classes: global experiments
that are aiming to detect the spectral signature of the cosmologically redshifted 21 cm transition
after averaging over the entire sky, and power spectrum experiments that incorporate angular
information to attempt to measure the spatial power spectrum of cosmological 21 cm perturbations.
Ongoing global experiments include EDGES \citep{2010Natur.468..796B, 2017ApJ...835...49M}, LEDA
\citep{todo_price_2017}, BIGHORNS \citep{2015PASA...32....4S}, SCI-HI \citep{2014ApJ...782L...9V},
and SARAS 2 \citep{2017arXiv170306647S}.  Ongoing power spectrum experiments include PAPER/HERA
\citep{2015ApJ...809...61A, 2016arXiv160607473D}, LOFAR \citep{2017ApJ...838...65P}, and the MWA
\citep{2016ApJ...833..102B}.

Just as for CMB experiments, foreground removal or suppression is an essential component of both
classes of 21 cm cosmology experiments. The brightness temperature of the galactic synchrotron
emission at high galactic latitudes is measured by \citet{2008AJ....136..641R} as
\begin{equation}
    T \sim 300\,{\rm K} \times \left(\frac{\nu}{150\,{\rm MHz}}\right)^{-2.5}\,.
\end{equation}
Therefore experiments conservatively need to achieve at least 4 orders of dynamic range against this
foreground emission before the cosmological signal can be measured. Current foreground removal
methods (for example, \citealt{2012ApJ...756..165P}) rely on the assumption that the foreground
emission (after convolving with the instrumental response) is spectrally smooth. However the
low-frequency radio sky is composed of several components: galactic synchrotron emission, free-free
emission and absorption, supernova remnants, radio galaxies, and a confusing background of point
sources.  Ideally a foreground removal strategy should be informed by the measured spatial and
frequency structure of all foreground components. However, this possibility is limited by the
availability of suitable high-fidelity low-frequency sky maps on angular scales ranging from tens of
degrees to arcminutes.

The Global Sky Model (GSM) \citep{2008MNRAS.388..247D, 2017MNRAS.464.3486Z} is a data-driven
interpolation of various maps between 10 MHz and 100 GHz. However the majority of information
contained in the GSM is derived at higher frequencies where the majority of the input maps are from.
Below 408 MHz, the interpolation is largely driven by the Haslam 408 MHz map
\citep{1981A&A...100..209H, 1982A&AS...47....1H}.  At lower frequencies, free-free absorption and
synchrotron self-absorption becomes increasingly important and hence spectral indices derived at
higher frequencies need corrections at lower frequencies.

Recently a host of new low-frequency sky surveys have been conducted including MSSS
\citep{2015A&A...582A.123H}, GLEAM \citep{2015PASA...32...25W}, and TGSS
\citep{2017A&A...598A..78I}. However, the primary data product generated by these surveys is a
catalog of radio sources. Surveys that capture the diffuse emission include at 45 MHz, where
\citet{2011A&A...525A.138G} produced a map of the sky with 5$^\circ$ resolution, and the LWA1 Low
Frequency Sky Survey \citet{2017MNRAS.469.4537D}, which covers a range of frequencies between 35 MHz
and 80 MHz with resolution betwen 4.5$^\circ$ and 2$^\circ$.

In this paper we will present a series of new low-frequency maps of the sky between 36.528 Mhz and
73.152 Mhz, capturing the full sky visible from the Owens Valley Radio Observatory (OVRO) with
angular resolution of roughly 10'. These maps are generated using $m$-mode analysis imaging -- a new
imaging technique for transit telescopes.

\section{All-Sky Imaging}

\begin{figure*}[ht]
    \plotone{maps/2017-04-20-rgb-map-smaller}
    \caption{
        This Mollweide-projected map is constructed from 3 maps of the sky at 41.760 MHz (red),
        57.456 MHz (green), and 73.152 MHz (blue) using data from the Owens Valley Long Wavelength
        Array. These maps were constructed using $m$-mode analysis imaging. 8 bright radio sources
        have been subtracted from the sky (Cyg A, Cas A, Vir A, Tau A, Hya A, Her A, Per B, and 3C
        353). The hole corresponds to declinations less than $-30^\circ$. Most of the diffuse
        emission is synchrotron, but the blue regions mottling the galactic plane are due to
        free-free absorption from \ion{H}{2} regions.
    }
    \label{fig:rgb}
\end{figure*}

The goal of all imaging algorithms is to estimate the brightness of the sky $I_\nu(\hat r)$ in the
direction $\hat r$ and frequency $\nu$. A radio interferometer measures the visibilities
$V_{ij,\nu}$ between pairs of antennas -- here numbered $i$ and $j$ respectively. If the antennas
are separated by the baseline $\vec b_{ij}$, and $A_\nu(\hat r)$ describes their response to the
incident radiation, then
\begin{equation}\label{eq:basic-imaging}
    V^{ij}_\nu = \int_\text{sky}
                 A_\nu(\hat r) I_\nu(\hat r)
                 \exp\bigg(2\pi i \hat r\cdot\vec b_{ij}/\lambda\bigg) \,\d\Omega \, .
\end{equation}
Imaging the output of a radio interferometer therefore consists of estimating $I_\nu(\hat r)$ given
the available measurements $V^{ij}_\nu$.

For later convenience we will define the baseline transfer function $B^{ij}_\nu(\hat r)$ such that
\begin{equation}\label{eq:baseline-transfer-function}
    V^{ij}_\nu = \int_\text{sky} B^{ij}_\nu(\hat r) I_\nu(\hat r) \,\d\Omega \, .
\end{equation}
The baseline transfer function defines the response of a single baseline to the sky, and is a
function of the antenna primary beam and baseline orientation.

Naively one might attempt to solve equation~\ref{eq:basic-imaging} by discretizing, and subsequently
solving the resulting matrix equation. If the interferometer is composed of $N_\text{base}$
baselines, and measures $N_\text{freq}$ frequency channels over $N_\text{time}$ integrations then
the entire data set consists of $N_\text{base}N_\text{freq}N_\text{time}$ complex numbers. If the
sky is discretized into $N_\text{pix}$ pixels then the relevant matrix has dimensions of
$(N_\text{base}N_\text{freq}N_\text{time})\times(N_\text{pix})$. For making single-channel maps with
the OVRO-LWA this becomes a 5 petabyte array (assuming each matrix element is a 64-bit complex
floating point number).  This matrix equation is therefore prohibitively large, and solving
equation~\ref{eq:basic-imaging} by means of discretization is usually impossible although
\citet{2017MNRAS.465.2901Z} demonstrate this technique with the MITEOR telescope.

Instead it is common to make mild assumptions that simplify equation~\ref{eq:basic-imaging} and ease
the computational burden in solving for $I_\nu(\hat r)$. For example, when all of the baselines
$\vec b_{ij}$ lie in a plane and the field-of-view is small, equation~\ref{eq:basic-imaging} can be
well-approximated by a two-dimensional Fourier transform \citep{2001isra.book.....T}. The
restriction on baseline co-planarity and field-of-view can be relaxed by using W-projection
\citep{2008ISTSP...2..647C}. Known primary beam effects can also be accounted for during imaging by
using A-projection \citep{2013ApJ...770...91B}.

\subsection{$m$-Mode Analysis}

On the other hand, transit telescopes can take advantage of a symmetry in
equation~\ref{eq:basic-imaging} that greatly reduces the amount of computer time required to image
the full-sky with exact incorporation of widefield imaging effects. This technique, called $m$-mode
analysis, also obviates the need for gridding, mosaicing, and multi-scale deconvolution. Instead the
entire sky is imaged in one coherent synthesis imaging step.

In this context we will define a transit telescope as any interferometer where the response pattern
of the individual elements does not change with respect to time. This may be an interferometer like
the OVRO-LWA where the correlation elements are fixed dipoles, but it may also be an interferometer
like LOFAR or the MWA if the steerable beams are held in a fixed position (not necessarily at
zenith).

We will briefly summarize $m$-mode analysis below, but the interested reader should consult
\citet{2014ApJ...781...57S, 2015PhRvD..91h3514S} for a complete derivation.

For a transit telescope, the visibilities $V^{ij}_\nu$ are a periodic function of sidereal
time.\footnote{
    This is not strictly true. Ionospheric fluctuations and non-sidereal sources (such as the sun)
    will violate this assumption. This paper will, however, demonstrate that the impact on the final
    maps is mild.
}
Therefore it is a very natural operation to compute the Fourier transform of the visibilities with
respect to sidereal time $\phi\in[0,2\pi)$.
\begin{equation}
    V^{ij}_{m,\nu} = \int_0^{2\pi} V^{ij}_\nu(\phi)\exp\bigg(-im\phi\bigg)\,\d\phi
\end{equation}
The output of this Fourier transform is the set of $m$-modes $V^{ij}_{m,\nu}$ where
$m=0,\,\pm1,\,\pm2,\,\ldots$ is the Fourier conjugate variable to the sidereal time. The $m$-mode
corresponding to $m=0$ is a simple average of the visibilities over sidereal time. Similarly $m=1$
corresponds to the component of the visibilities that varies over half-day timescales. Larger values
of $m$ correspond to components that vary on quicker timescales.

It can be shown that there is a discrete linear relationship between the $m$-modes $V^{ij}_{m,\nu}$
and the spherical harmonic coefficients of the sky brightness $a_{lm,\nu}$.
\begin{equation}\label{eq:m-mode-sum-equation}
    V^{ij}_{m,\nu} = \sum_l B^{ij}_{lm,\nu} a_{lm,\nu}\,,
\end{equation}
where the transfer coefficients $B^{ij}_{lm,\nu}$ define the interferometer's response to the sky.
For example, the transfer coefficients are a function of the baseline and antenna primary beam
pattern.

Equation~\ref{eq:m-mode-sum-equation} can be recognized as a matrix equation where the transfer
matrix $\b B$ is block-diagonal.
\begin{equation}
    \b B = \left(\begin{array}{cccc}
        m = 0 &&& \\
              & m=\pm1 && \\
              && m=\pm2 & \\
              &&& \ddots \\
    \end{array}\right)
\end{equation}
The vector $\b v$ contains the list of $m$-modes and the vector $\b a$ contains the list of
spherical harmonic coefficients representing the sky brightness. In order to take advantage of the
block-diagonal structure in $\b B$, $\b v$ and $\b a$ must be sorted by the value of $m$. Positive
and negative values of $m$ are grouped together because the brightness of the sky is real-valued,
and the spherical harmonic transform of a real-valued function has $a_{l(-m)} = (-1)^m a_{lm}^*$.
\begin{equation}\label{eq:m-mode-matrix-equation}
    \overbrace{\left(
        \begin{array}{c}
            \vdots \\
            m\text{-modes} \\
            \vdots \\
        \end{array}
    \right)}^{\b v}
    =
    \overbrace{\left(
        \begin{array}{ccc}
            \ddots & & \\
            & \text{transfer matrix} & \\
            & & \ddots \\
        \end{array}
    \right)}^{\b B}
    \overbrace{\left(
        \begin{array}{c}
            \vdots \\
            a_{lm} \\
            \vdots \\
        \end{array}
    \right)}^{\b a}
\end{equation}

In practice we now need to pick the set of spherical harmonics we will use to represent the sky. For
an interferometer like the OVRO-LWA with lots of short baselines, a sensible choice is to use all
spherical harmonics with $l\le l_\text{max}$ for some $l_\text{max}$. The parameter $l_\text{max}$
is determined by the maximum baseline length of the interferometer.  For an interferometer without
short spacings, a minimum value for $l$ might also be used. This $l_\text{min}$ parameter should be
determined by the minimum baseline length.  When creating the maps presented in this paper, we use
$l_\text{min} = 0$ and $l_\text{max} = 1000$.

The interferometer's sensitivity to $l=0$, however, deserves special consideration.
\citet{2016ApJ...826..116V} prove -- under fairly general assumptions -- that an interferometer is
only sensitive to the monopole of the sky brightness if there exists some form of cross-talk or
common-mode noise. In fact, the sensitivity of the interferometer is proportional to the amplitude
of these effects. For consistency we will include $a_{00}$ while solving
Equation~\ref{eq:m-mode-matrix-equation} for the vector $\b a$, but we set $a_{00} = 0$ afterwards
because we do not have a measurement of the amplitude of cross-talk or common-mode noise, which
would be needed to correctly calibrate $a_{00}$.

The size of a typical block in the transfer matrix is
$(2N_\text{base}N_\text{freq})\times(l_\text{max})$. If each element of the matrix is stored as a
64-bit complex floating point number, a single block is 500 MB for the case of single-channel
imaging with the OVRO-LWA. Compare this number with the 5 PB required for the naive approach.  The
power of $m$-mode analysis is the block-diagonal structure of
equation~\ref{eq:m-mode-matrix-equation}.  By breaking up the problem into $N$ independent blocks,
the computational complexity involved in inverting the equation is reduced by a factor $N^3$. For
the case of the OVRO-LWA the equation breaks up into $\sim10^3$ blocks and so we save a factor of
$\sim10^9$ in processing time by using $m$-mode analysis.

\subsection{$m$-Mode Analysis Imaging}

Imaging in $m$-mode analysis essentially amounts to inverting
equation~\ref{eq:m-mode-matrix-equation} to solve for the spherical harmonic coefficients $\b a$.
The linear-least squares solution, which minimizes $\|\b v - \b B\b a\|^2$, is given by
\begin{equation}
    \b{\hat a}_\text{LLS} = (\b B^*\b B)^{-1}\b B^*\b v\,,
\end{equation}
where $^*$ indicates the conjugate-transpose. However, usually one will find that $\b B$ is not
full-rank and hence $\b B^*\b B$ is not an invertible matrix. For example, an interferometer located
in the northern hemisphere will never see a region of the southern sky centered on the southern
celestial pole. The $m$-modes contained in the vector $\b v$ must contain no information
about the sky around the southern celestial pole, and therefore the act of multiplying by $\b B$
must destroy some information about the sky. The consequence of this fact is that $\b B$ must have
at least one singular value that is equal to zero. It is then trivial to show that $\b B^*\b B$ must
have at least one eigenvalue that is equal to zero, which means it is not an invertible matrix.

Another way of looking at the problem is that because the interferometer is not sensitive to part of
the southern hemisphere, there are infinitely many possible solutions to
equation~\ref{eq:m-mode-matrix-equation} that will fit the measured data equally well. So we need to
regularize the problem and apply an additional constraint that prefers a unique solution. For
example, you may prefer that in the absence of any information the sky should be set to zero or you
may prefer that the sky should be set to some prior expectation.

\subsubsection{Tikhonov Regularization}

The process of Tikhonov regularization minimizes $\|\b v - \b B\b a\|^2 + \varepsilon\|\b a\|^2$ for
some arbitrary value of $\varepsilon > 0$ chosen by the observer. The solution that minimizes this
expression is given by
\begin{equation}\label{eq:tikhonov-solution}
    \atikh = (\b B^*\b B + \varepsilon\b I)^{-1}\b B^*\b v\,.
\end{equation}
Tikhonov regularization adds a small value $\varepsilon$ to the diagonal of $\b B^*\b B$, fixing the
matrix's singularity. We can see this by using the singular value decomposition (SVD) of the matrix
$\b B = \b U \b \Sigma \b V^*$. Equation~\ref{eq:tikhonov-solution} becomes
\begin{equation}
    \atikh = \b V (\b\Sigma^2 + \varepsilon \b I)^{-1}\b\Sigma \b U^*\b v\,,
\end{equation}
where
\[
    \b\Sigma = \left(
        \begin{array}{ccc}
            \sigma_1 & & \\
                     & \sigma_2 & \\
                     & & \ddots \\
        \end{array}
    \right)\,.
\]
The diagonal elements of $\b\Sigma$ are the singular values of $\b B$ and so the contribution of
each singular component to the Tikhonov-regularized solution is scaled by $\sigma_i / (\sigma_i^2 +
\varepsilon)$, where $\sigma_i$ is the singular value for the $i$th singular component. Tikhonov
regularization therefore acts to suppress any component for which
$\sigma_i\lesssim\sqrt{\varepsilon}$.  If $\sigma_i = 0$, the component is set to zero.

In practice the measurement $\b v$ is corrupted by noise with covariance $\b N$. For illustrative
purposes we will assume that $\b N=n\b I$ for some scalar $n$. In this case the covariance of the
Tikhonov-regularized spherical harmonic coefficients is
\begin{equation}
    \b C = n \b V (\b\Sigma^2 + \varepsilon\b I)^{-2} \b\Sigma^2 \b V^*\,.
\end{equation}
Each singular component is scaled by a factor of $\sigma_i^2/(\sigma_i^2 + \varepsilon)^2$.  In the
absence of Tikhonov regularization ($\varepsilon=0$) singular components with the smallest singular
values -- the ones that the interferometer is the least sensitive to -- actually come to dominate
the covariance of the measured spherical harmonic coefficients. Tikhonov regularization improves
this situation by downweighting these components.

While Tikhonov regularization will force unmeasured modes to zero, if a prior map of the sky already
exists, it will be preferrable to instead minimize $\|\b v - \b B\b a\|^2 + \varepsilon \|\b a-\b
a_\text{prior}\|^2$.

\subsubsection{L-Curves}

\begin{figure}[t]
    \plotone{figures/lcurve/lcurve}
    \caption{
        An example L-curve using OVRO-LWA data at 36.528 MHz. The $x$-axis is the norm of the
        solution (in this case the spherical harmonic coefficients) given in arbitrary units. The
        $y$-axis is the least-squares norm given in arbitrary units. $\varepsilon$ is the
        regularization parameter. When the regularization parameter is small, the norm of the
        solution grows rapidly. When the regularization parameter is large, the least-squares norm
        grows rapidly. The optimal value for the regularization parameter is near
        $\varepsilon=0.01$.
    }
    \label{fig:lcurve}
\end{figure}

Tikhonov regularization requires the observer to pick the value of $\varepsilon$. If $\varepsilon$
is too large then too much importance is placed on minimizing the norm of the solution and the
least-squares residuals will suffer. However if $\varepsilon$ is too small then the problem will be
poorly regularized and the resulting sky map may not represent the true sky. Picking the value of
$\varepsilon$ therefore requires understanding the trade-off between the two norms.

This trade-off can be analyzed quantitatively by trialing several values of $\varepsilon$ and
computing $\|\b v - \b B\b a\|^2$ and $\|\b a\|^2$ for each trial. An example is shown in
Figure~\ref{fig:lcurve}. The shape of this curve has a characteristic L-shape, and as a result this
type of plot is called an L-curve. The ideal value of $\varepsilon$ lies near the turning point of
the plot. At this point a small decrease in $\varepsilon$ will lead to an undesired rapid increase
in $\|\b a\|^2$, and a small increase in $\varepsilon$ will lead to an undesired rapid increase in
$\|\b v - \b B\b a\|^2$.

In practice, the L-curve should be used as a guide to estimate a reasonable value of $\varepsilon$.
However better results can often be obtained by tuning the value of $\varepsilon$. For instance
increasing the value of $\varepsilon$ can improve the noise properties of the map by down-weighting
noisy modes. Decreasing the value of $\varepsilon$ can improve the resolution in the map by
up-weighting the contribution of longer baselines, which are likely fewer in number. In this respect
choosing the value of $\varepsilon$ is analagous to picking the weighting scheme in traditional
imaging where robust weighting schemes can be tuned to similar effect \citep{briggs}.

\subsubsection{The Moore-Penrose Pseudoinverse}

The Moore-Penrose pseudoinverse (denoted in this paper with a superscript $\dagger$), is commonly
applied to find the minimum-norm linear-least squares solution to a set of linear equations. This
can be used in place of Tikhonov regularization as
\begin{equation}
    \b{\hat a}_\text{Moore-Penrose} = \b B^\dagger\b v\,.
\end{equation}
Much like Tikhonov regularization, the Moore-Penrose pseudoinverse sets components with small
singular values (below some user-defined threshold) to zero. Components with large singular values
(above the user-defined threshold) are included in the calculation at their full amplitude with no
down-weighting of modes near the threshold. The essential difference between using the Moore-Penrose
pseudoinverse and Tikhonov regularization is that the pseudoinverse defines a hard transition from
on to off. Modes are either set to zero or included in the map at their full amplitude. On the other
hand Tikhonov regularization smoothly interpolates between these behaviors. Because of this,
Tikhonov regularization tends to produce better results in practical applications.

\subsection{$m$-Mode Analysis Image Deconvolution}

$m$-mode analysis imaging automatically deconvolves angular scales described by modes for which the
singular values of $\b B$ are $\gg \sqrt{\varepsilon}$. For example, structures on angular scales
smaller than $\lambda/b_\text{min}$ and larger than $\lambda/b_\text{max}$ (where $b_\text{min}$ and
$b_\text{max}$ are the minimum and maximum baseline lengths) are generally well-measured by the
interferometer and will be automatically deconvolved during the imaging process. However, as the
total integration time of the data set increases, the noise in the data decreases. One consequence
of this is that the regularization parameter $\varepsilon$ can be decreased. This increases the
number of modes for which $\sigma_i \gg \sqrt{\varepsilon}$, and as a consequence the automatic
deconvolution can extend to scales larger than $\lambda/b_\text{min}$ and smaller than
$\lambda/b_\text{max}$.

However point sources do not get deconvolved by the imaging process even in the limit of infinite
SNR and $\varepsilon \rightarrow 0$. This is because point sources carry power to large values of
$l$ and by computational necessity we truncated the spherical harmonic representation of the sky at
some $l_\text{max}$. A consequence of this is that point sources will appear in the maps with a
characteristic point spread function (PSF). The PSF may be computed with:
\begin{equation}
    \b a_\text{PSF}(\theta, \phi)
        = (\b B^*\b B + \varepsilon\b I)^{-1}\b B^*\b B\b a_\text{PS}(\theta, \phi)\,,
\end{equation}
where $\b a_\text{PSF}(\theta, \phi)$ is the vector of spherical harmonic coefficients representing
the PSF at the spherical coordinates $(\theta, \phi)$, and $\b a_\text{PS}(\theta, \phi)$ is the
vector of spherical harmonic coefficients for a point source at $(\theta, \phi)$ given by
\begin{equation}
    a_{lm, \text{PS}}(\theta, \phi) = Y_{lm}^*(\theta, \phi)\,.
\end{equation}

In general the PSF can be a function of the right ascension and declination. However point sources
at the same declination take the same track through the sky and therefore (barring any ionospheric
effects) will have the same PSF. The PSF is actually only a function of the declination. For
example, sources at low elevations will tend to have an extended PSF along the north-south axis due
to baseline foreshortening. An example computed PSF is shown in Figure~\ref{fig:psf}.

The computed PSF as a function of frequency and declination can be used to adapt the CLEAN algorithm
\citep{1974A&AS...15..417H} to $m$-mode analysis.

\begin{algorithm}
    \KwData{dirty maps generated from the output of Equation~\ref{eq:tikhonov-solution};}
    \KwResult{clean maps with point sources deconvolved;}
    pre-compute $\b M = (\b B^*\b B + \varepsilon\b I)^{-1}\b B^*\b B$\;
    \While{noise in map $>$ threshold}{
        identify the pixel with the largest amplitude\;
        compute $\b a_\text{PSF} = \b M \b a_\text{PS}$\;
        update $\b a \leftarrow \b a - (\textit{gain})\times\b a_\text{PSF}$\;
    }
    convolve removed components with a CLEAN-beam and restore to the map\;
\end{algorithm}

It should be noted that for interferometers with more baselines than the number of spherical
harmonics used in the maps, $\b M$ is a much smaller matrix than the full transfer matrix $\b B$.
Therefore the cost of multiplying by $\b M$ can actually be dominated by the cost of computing the
spherical harmonics describing a point source at the identified pixel.












\section{Observations}

\subsection{The Owens Valley Long Wavelength Array}

\begin{figure}[t]
    \plotone{figures/antenna-layout/antenna-layout}
    \caption{
        This figure shows the antenna layout for the OVRO-LWA. Black dots correspond to antennas
        within the 200 m diameter core of the array. The 32 open circles are the expansion antennas
        built in early 2016 in order to increase the longest baseline to 1.5 km. The 5 crosses are
        antennas equipped with noise-switched front ends.
    }
    \label{fig:antenna-layout}
\end{figure}

\begin{figure}[t]
    \plotone{figures/psf/spw04-psf-45-degrees}
    \caption{
        The computed PSF for the OVRO-LWA for a point source at a declination of $+45^\circ$ and
        frequency of $36.528$ MHz.
    }
    \label{fig:psf}
\end{figure}

The Owens Valley Long Wavelength Array (OVRO-LWA) is a 288-element interferometer located at the
Owens Valley Radio Observatory (OVRO) near Big Pine, California \citep{todo_hallinan_2017}.  The
OVRO-LWA is a low-frequency instrument with instantaneous bandwidth covering 27.384 MHz to 84.912
MHz and 24 kHz channelization.  Each antenna stand hosts two perpendicular broadband dipoles so that
there are $288\times2$ signal paths in total. These signal paths feed into the 512-input LEDA
correlator \citep{2015JAI.....450003K}, which allows the OVRO-LWA to capture the entire visible
hemisphere in a single snapshot image.  In the current configuration 32 antennas (64 signal paths)
are unused.

The 288 antennas are arranged in a pseudo-random configuration optimized to minimize sidelobes in
snapshot imaging.  251 of the antennas are contained within a 200 m diameter core. 32 antennas are
placed outside of the core in order to extend the maximum baseline length out to $\sim$1.5 km. The
final 5 antennas are equipped with noise-switched front ends for calibrated total power measurements
of the global sky brightness.  These antennas are used as part of the LEDA experiment
\citep{todo_price_2017} to measure the global signal of 21 cm absorption from the cosmic dawn.
Figure~\ref{fig:antenna-layout} is a diagram of the antenna configuration.

Beginning at 2017-02-17 12:00:00 UTC time, 28 consecutive hours of data was collected. This time was
chosen based on the fact that it was raining at OVRO and rain tends to improve the low-frequency RFI
environment considerably. During this time the OVRO-LWA operated as a zenith-pointing drift scanning
interferometer.  The correlator dump time was selected to be 13 seconds such that the correlator
output evenly divides a sidereal day.

The geomagnetic conditions during this time were mild. The Disturbance storm time (Dst) index was
$>-30$ nT during the entirety of the observing period.\footnote{
    The Dst index was obtained from the World Data Center for Geomagnetism, Kyoto University
    (\url{http://swdcwww.kugi.kyoto-u.ac.jp/}).
}
Following the classification scheme of \citet{2008GMS...181.....K}, a weak geomagnetic storm has
$\text{Dst} < -30$ nT. Therefore there were no geomagnetic storms during the time of these
observations.

\subsection{Gain Calibration}

Antenna gain calibration is accomplished using an iterative method independently developed by
\citet{2008ISTSP...2..707M} and \citet{2014A&A...571A..97S}. The calibration routine is written in
the Julia programming language \citep{doi:10.1137/141000671}, and is publicly available
online\footnote{\url{https://github.com/mweastwood/TTCal.jl}} under an open source license (GPLv3+).

The antenna complex gains are measured from a 7.2 hour track of data from when Cyg A and Cas are at
high elevations. The gains measured in this way are then used to calibrate the entire 28 hour
dataset. Cyg A and Cas A are -- by an order of magnitude -- the brightest point-like radio sources
in the northern hemisphere. Therefore the optimal time to solve for the interferometer's gain
calibration is when these sources are at high elevations. \todo{cite some statistics on how good
this calibration is? what do I quantify?}

\todo{at the moment we calibrated with the Baars spectrum for Cas. we should actually iterate once
and use the new spectrum that comes out of the beam fitting routine}

The flux scale is tied to the work of \citet{1977A&A....61...99B, 2012MNRAS.423L..30S,
2017ApJS..230....7P}.

Temperature fluctuations within the electronics shelter generate 0.1 dB sawtooth oscillations in the
analog gain. These oscillations occur with a variable 15 to 17 minute period. The amplitude of these
gain fluctuations is calibrated by smoothing the autocorrelation amplitudes on 45 minute timescales.
The ratio of the measured auto-correlation power to the smoothed auto-correlation power defines a
per-antenna amplitude correction that is then applied to the cross-correlations.

\todo{tuesday}

\subsection{Primary Beam Measurements}

\begin{figure*}[ht]
    \plottwo{figures/beam/source-tracks}{figures/beam/stokes-I-beam}
    \caption{
        Example Stokes-I beam model (from spw04).
    }
    \label{fig:beam}
\end{figure*}

In order to generate wide-field images of the sky, the response of the antenna to the sky must be
known. Fortunately drift-scanning interferometers like the OVRO-LWA can empirically measure their
primary beam under a mild set of symmetry assumptions \citep{2012AJ....143...53P}. In this work we
assume that the primary beam is invariant under north-south and east-west flips, and additionally
that the $x$- and $y$-dipoles have the same response to the sky after rotating one by 90$^\circ$.
These symmetries are apparent in the antenna design, but real-world defects and coupling with nearby
antennas will contribute towards breaking these symmetries at some level. However some amount of
symmetry must be assumed in order to break the degeneracy between source flux and beam amplitude
when the flux of a source is unknown.

We measure the flux of several bright sources (Cyg A, Cas A, Tau A, Vir A, Her A, Hya A, Per B, and
3C 353) as they pass through the sky and then fit a beam model composed of Zernike polynomials to
those flux measurements. We select the basis functions to have the desired symmetry ($Z_0^0$,
$Z_2^0$, $Z_4^0$, $Z_4^4$, $Z_6^0$, $Z_6^4$, $Z_8^0$, $Z_8^4$, $Z_8^8$) and the beam amplitude at
zenith is constrained to be unity. See Figure~\ref{fig:beam} for an illustration of the fitted beam
model.

\subsection{Source Removal}

\subsubsection{Cassiopeia A and Cygnus A}

Without removing bright sources from the data, sidelobes from bright sources will dominate the
variance in the image.  At 74 MHz Cyg A is a 15,000 Jy source \citep{2017ApJS..230....7P}. A
conservative estimate for the confusion limit at 74 MHz with a 10 arcminute beam is 500 mJy
\citep{2012RaSc...47.0K04L}. Therefore we require that Cyg A's sidelobes be at most of $-$45 dB down
from its peak flux to prevent Cyg A's sidelobes from dominating the variance in the image.
\todo{Measure Cyg A's sidelobe levels if it is not subtracted}

At low frequencies, propagation effects through the ionosphere must be accounted for in order to
achieve high dynamic range images. This necessitates the use of direction-dependent calibration and
peeling \citep{2008ISTSP...2..707M, 2015MNRAS.449.2668S}.  In the dataset used in this paper,
scintillation and diffraction events on the timescale of a single integration (13 seconds) are
observed. Therefore the direction dependent calibration changes on these timescales and the we must
solve for one set of complex gains per source per integration.

The largest angular scale of Cas A is $\sim$8 arcminutes, while the largest angular scale of Cyg A
is $\sim$2 arcminutes. With a 10 arcminute resolution, the OVRO-LWA marginally resolves both
sources. A resolved source model is needed for both sources. We fit a self-consistent resolved
source model to eachs ource. This is performed by minimizing the minimizing the variance within an
aperture located on each source after peeling. By phasing up a large number of integrations before
imaging (at least over 1 hour) it is possible to smear out the contribution of the rest of the sky.
We then use NLopt's Sbplx routine \citep{nlopt, sbplx} to vary the parameters in a source model
until the variance within the aperture is minimized.

\todo{Show a before and after with the updated source models}

\todo{Give a plot with the updated source models}

\subsubsection{The Sun}

The Sun can be trivially removed from any map of the sky by constructing the map using only data
taken at night. A map of the entire sky can be obtained by using observations spaced 6 months apart.
However the dataset used in this paper consists of 28 consecutive hours. Additionally while most
astronomical sources at these frequencies have negative spectral indices, the Sun has a large
positive spectral index. Therefore more care will need to be taken in subtracting the Sun at higher
frequencies than at lower frequencies.

\todo{describe sun removal -- this isn't entirely settled yet}

\section{Error Analysis}

\subsection{Terrestrial Interference and Pickup}

\begin{figure*}[ht]
    \gridline{
        \fig{figures/smeared-images/before-component-removal-spw14}{0.22\textwidth}{(a)}
        \fig{figures/smeared-images/after-component-removal-spw14}{0.22\textwidth}{(b)}
        \fig{figures/smeared-images/rfi-like-component-spw14}{0.22\textwidth}{(c)}
        \fig{figures/smeared-images/pickup-like-component-spw14}{0.22\textwidth}{(d)}
    }
    \caption{
        This figure illustrates the process of fitting for terrestrial sources of correlated noise.
        Each image is of the entire visible hemisphere above the OVRO-LWA but the data has been
        averaged over the entire 28 hour observing period. Because of this Cas A and Cyg A are
        smeared along tracks of constant declination. These images are generated using WSCLEAN
        \citep{2014MNRAS.444..606O}, uniform weighting, and baselines shorter than 15 wavelengths
        are flagged. \textbf{(a)} This image shows the initial data before any sources have been
        removed. \textbf{(b)} This image shows the final state after 3 terrestrial sources of
        correlated noise have been removed. \textbf{(c)} This image illustrates the contribution of
        an RFI source towards the north-west that was removed by peeling. \textbf{(d)} This removed
        component is not associated with a dot on the horizon.  Instead it is likely associated with
        common-mode pickup in the analog signal chain.
    }
    \label{fig:fitrfi}
\end{figure*}


When writing down equation~\ref{eq:basic-imaging}, it is implicitly assumed that the correlated
voltage fluctuations measured between pairs of antennas are exclusively generated by astronomical
sources of radio emission. In practice this implicit assumption can be violated. For instance a
low-frequency interferometer located in the vicinity of an arcing power line will see an additional
contribution from the radio-frequency interference (RFI) generated by the arcing process. Similarly
common-mode pickup along the analog signal path of the interferometer may generate an additional
spurious contribution to the measured visibilities. While the amplitude and phase of these
contaminating signals may fluctuate with time, they do not sweep across the sky at the sidereal rate
characteristic of astronomical sources.

The Owens Valley is an important source of water and power for the city of Los Angeles.
Unfortunately this means that high voltage power lines run along the valley to the west of the OVRO-
LWA. Some of these power line poles have faulty insulators that arc and produce pulsed, broadband
RFI. Because these poles exist in the near-field of the array, we have been able to localize some of
them by using the curvature of the incoming wavefront to infer a distance. Efforts are currently
underway to work with the utility pole owners to have these insulators replaced.

In the meantime it is possible to suppress their contamination in the dataset. The contribution of
these RFI sources to the visibilities can be plainly seen by averaging $>24$ hours of data with the
phase center set to zenith. In this way, true sky components are smeared along tracks of constant
declination while terrestrial sources (ie. the arcing power lines or any common-mode pickup) are not
smeared.  Obtaining a model for the RFI is complicated by the fact that the contaminating sources
are at extremely low elevations where the antenna response is essentially unknown (and inhomogeneous
due to antenna shadowing effects). Therefore it is not enough to know the physical location of the
faulty insulator generating the RFI. In addition you must also know the response of each antenna
(amplitude and phase) in the appropriate direction. This motivates the use of peeling
\citep{2008ISTSP...2..707M, 2015MNRAS.449.2668S}, which allows the antenna response to be a free
parameter.  Therefore model visibilities for the RFI can be obtained by peeling the sources after
smearing the visibilities over $>24$ hours. See Figure~\ref{fig:fitrfi} for an illustration of this
process.

However while attempting to peel RFI sources from the averaged visibilities, it was discovered that
frequently peeling would prefer to remove components from the visibilities that are not obviously
associated with any source on the horizon or elsewhere in the sky (see panel (d) in
Figure~\ref{fig:fitrfi}). In many cases the amplitude of these unassociated components exhibit the
same sawtooth oscillations indicative of gain fluctuations in the analog electronics even after
these gain fluctuations have been calibrated. This seems to imply that that these unassociated
components originate as common-mode pickup in the analog signal chain. That is, if the pickup occurs
somewhere in the middle of the analog signal chain, it will not see the same gain as the
astronomical signal. Therefore correcting the temperature dependence of the gains for the
astronomical signal does not correct the temperature dependence of the pick-up and hence it will
have a sawtooth pattern with respect to time.

The first step in equation~\ref{eq:tikhonov-solution} is to compute $\b B^*\b v$. In this step we
compute the projection of the measurement $\b v$ onto the space spanned by the columns of $\b B$.
Each column of $\b B$ describes the interferometer's response to a corresponding spherical harmonic
coefficient of the sky brightness distribution. Therefore the act of computing $\b B^*\b v$ is to
project the measured $m$-modes onto the space of $m$-modes which could be generated by astronomical
sources. The degree to which a source of terrestrial interferer will contaminate a map generated
using $m$-mode analysis imaging is determined by its amplitude after projection.

For instance, a bright interfering source might contribute $\b v_\text{terrestrial}$ to the measured
$m$-modes. However, if $\b v_\text{terrestrial}$ is actually perpendicular to all of the columns of
$\b B$, there will be no contamination in the map because $\b B^*\b v_\text{terrestrial} = \b 0$.
In practice this is unlikely. In general the contamination is proportional to the overall amplitude
of the interference ($\|\b v_\text{terrestrial}\|$) and the degree to which the interference mimics
an astronomical signal ($\|\b B^*\b v_\text{terrestrial}\|/\|\b v_\text{terrestrial}\|$).

\todo{compute B*v for panels (c) and (d) to show that the pick-up is pretty bad. my expectation is
that it is actually worse than the RFI, but I don't actually know for sure}

These terrestrial sources do not rotate with the sky and hence their contamination tends to be
restricted to modes with small $m$. In this dataset the contamination is largely restricted to $m
\lesssim 3$. As a result if the contamination is not suppressed, it will manifest itself as rings
along stripes of constant declination. This effect is plainly visible in Figure \todo{make it}.

However because these rings are so distinctive in the final maps, it is possible to construct a
Wiener filter that removes them at the cost of losing a small amount of information about
astronomical sources.

\todo{describe how the Wiener filter works}

\subsection{Beam Errors}

A good model of the antenna beam is essential for widefield imaging. Because $m$-mode analysis
imaging operates on a full sidereal day of data, images are constructed after watching each point in
the sky move through large slices through the beam (excepting the celestial poles). The beam model
therefore serves two purposes:

\begin{enumerate}
    \item setting the flux scale as a function of declination
    \item reconciling observations from two separate sidereal times
\end{enumerate}

In the first case, all sources at a given declination take the same path through the antenna primary
beam. If the antenna response is overestimated along this track then all sources at this declination
will have their flux underestimated. Similarly if the antenna response is underestimated then all
the sources will have their flux overestimated.

\todo{discuss the possibility of declination dependent flux-scale errors in these maps}

The second case is more subtle. Sources are observed at a wide range of locations in the primary
beam of the antenna. The imaging process must reconcile all of these observations together. The beam
model essentially provides the instructions on how to do this. For example if at a time $t_1$ the
antenna gain towards a source is $g$ and at a later time $t_2$ the antenna gain towards that same
source (which has now moved) is $g/2$, then in order to correctly estimate the flux of the source
the observations from time $t_1$ need to be multiplied by $1/g$ but the observations from time $t_2$
need to be multiplied by $2/g$. A mistake in the beam model here will lead to a mistake in the
estimated flux.  However in general the amplitude and phase of the antenna response needs to be
known and so beam model errors can lead to observations being combined with incorrect phase as well.
This will lead to a degraded PSF in the final map. \todo{quick simulation of this?}

\subsection{The Ionosphere}

One of the key assumptions made by $m$-mode analysis is that the sky is completely static.  We
assume that the only time-dependent behavior is the rotation of the Earth, which slowly rotates the
sky through the fringe patterns of the interferometer. At low frequencies the ionosphere violates
this assumption. In particular, ionospheric scintillation will cause even static sources to exhibit
significant variability.

The correlation observed on a given baseline for a single point source is
\begin{equation}
    V_\nu(t_{\textrm{sidereal}}) = I_\nu B_\nu(t_{\textrm{sidereal}}),
\end{equation}
where $I_\nu$ is the flux of the source at the frequency $\nu$, and $B_\nu$ is the baseline transfer
function. The transfer function is a function of the direction to the source, which is in turn a
function of the sidereal time $t_{\textrm{sidereal}}$. If the source is varying, from intrinsic
variability or due to scintillation, than the source flux is also a function of the time coordinate
$t$ such that
\begin{equation}
    V_\nu(t_{\textrm{sidereal}}) = I_\nu(t) B_\nu(t_{\textrm{sidereal}}),
\end{equation}
where $t_{\textrm{sidereal}} = (t \mod 23.9345\,\textrm{hours})$.

Now assume we have observed with our interferometer for a single sidereal day.  In order to compute
the $m$-modes we must Fourier transform with respect to sidereal time. In a real measurement this is
a discrete Fourier transform of the observed correlation with respect to time (where the sum over
time is restricted to a sidereal day). \todo{fix normalization}
\begin{equation}
    V_{\nu, m} = \sum_{t} V_\nu(t) e^{-imt}
\end{equation}
Define $V_{\nu, m}^{\textrm{static}}$ to be the observed $m$-modes if the source was actually static
($I_\nu(t) \equiv I_{\nu,0}$). Then as a consequence of the Fourier convolution theorem
\begin{equation}
    V_{\nu, m} = \sum_{m^\prime} V_{m^\prime}^\textrm{static} I_{\nu, m-m^\prime}.
\end{equation}

This will tend to scatter power between $m$-modes. \todo{cite richard for this}

\todo{Simulate images of point sources with this}

\todo{I think the conclusion is sort of exactly what you would expect: refraction broadens the PSF
while scintillation effects the flux scale, but honestly not by much because we have such large
averages that unless the scintillation is Cauchy distributed it is not that bad..}

\subsection{Thermal Noise}

\todo{describe jack-knife tests}

\section{Results}

\subsection{Sky Maps}

\floattable
\begin{deluxetable}{ccccc}
    \tablecaption{Summary of the produced maps\label{tab:summary}}
    \tablehead{
        \colhead{\#} & \colhead{$\nu$ / MHz} & \colhead{$\Delta\nu$ / MHz} &
        \colhead{$\theta$ / arcmin} & \colhead{noise / Jy}
    }
    \startdata
        1 & 36.528 & 0.024 & & \\
        2 & 41.760 & 0.024 & & \\
        3 & 46.992 & 0.024 & & \\
        4 & 52.224 & 0.024 & & \\
        5 & 57.456 & 0.024 & & \\
        6 & 62.688 & 0.024 & & \\
        7 & 67.920 & 0.024 & & \\
        8 & 73.152 & 0.024 & & \\
    \enddata
\end{deluxetable}

\subsection{Error Maps}


\section{Discussion}

\citet{todo_mishra_2017}

\todo{Wednesday}

\bibliographystyle{apj}
\bibliography{paper}

\end{document}

