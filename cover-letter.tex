\documentclass{article}

\usepackage{hyperref}
\usepackage{color}
\newcommand{\revision}[1]{\textcolor{red}{\textbf{#1}}}

\begin{document}

\section{Cover Letter}

We appreciated the reviewer's helpful comments, particularly the concern about ionospheric
scintillation resolving to antenna gains.  We interpreted this concern as primarily applicable to
direction-dependent calibration, which is used during the removal of bright point sources (primarily
Cyg A and Cas A).

The relevant length scales in the ionosphere are the diffractive scale (essentially the length over
which the ionosphere imparts a 1 radian phase difference) and the Fresnel scale. In the usual case
where the diffractive scale is much larger than the Fresnel scale, you can see phase and amplitude
scintillation, but these effects can be absorbed into direction-dependent complex gains. However,
when the diffractive scale is less than the Fresnel scale, you can see a point source break apart
into multiple images. This cannot be entirely absorbed into direction-dependent complex gains, but
the error you make during source removal leads to residual artifacts in the vicinity of the source.
We added additional text to Section 3.5.1 that addresses this possibility.

The second-to-last paragraph in Section 2.1 explains why we restricted ourselves to single channel
imaging (reproduced below):

\begin{quote}
    If each element of the matrix is stored as a 64-bit complex floating point number, a single
    block is 500 MB for the case of single-channel imaging with the OVRO-LWA, which a modern
    computer can easily store and manipulate in memory.  However, with additional bandwidth these
    blocks quickly become unwieldy; thus as a first pass, the analysis in this paper is restricted
    to single-channel imaging.
\end{quote}

However, additional text and explicit runtimes have been added to clarify how additional bandwidth
impacts the amount of processing time necessary to generate the maps presented in this work.

Changes to the text are highlighted with \revision{red font and boldface}. These changes are listed
below.


\section{List of Changes}

\begin{enumerate}
    \item The "Owens Valley Radio Observatory Long Wavelength Array" (OVRO-LWA) was incorrectly
        referred to as the "Owens Valley Long Wavelength Array". This has been corrected throughout
        the article.

    \item A URL pointing to where the final sky maps can be downloaded from LAMBDA has been added
        (\url{https://lambda.gsfc.nasa.gov/product/foreground/fg_ovrolwa_radio_maps_info.cfm}).

    \item In Section 1, the brightness temperature of the galactic synchrotron emission was updated
        to reference a more recent measurement.

    \item In Section 2.1, additional text clarifies the use of multiple ``N''s and gives some
        explicit examples typical runtime for each step of the analysis. There is also an additional
        sentence mentioning that for a block diagonal matrices, all of the blocks can be manipulated
        independently.

    \item In Section 3.5.1 we added additional text discussing the impact of the ionosphere in the
        strong scattering regime (summarized above).

    \item In Section 4, the text incorrectly mentioned estimating the variance with a jackknife
        estimator.  We are actually estimating the standard error, and the standard error is
        reported in Table 1.

    \item The original text described a $\sim$45 arcmin rotation that was apparent in comparisons
        with the LWA1 Low Frequency Sky Survey. This rotation was confirmed to originate from a
        rotation in the LWA1's antenna coordinates, and has been fixed.  Section 4.1.1 has been
        updated accordingly, and Figure 13 has also been updated to provide comparisons with the
        corrected LWA1 maps.
\end{enumerate}

\end{document}

